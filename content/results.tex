\section{Results}

\subsection{Overall research output}
Since the release of the \textit{Nudge}, by Thaler and Sunstein in 2008, the concept of nudges gains increasing interest in several research streams and domains. Table \ref{table:research-output} gives an overview of the overall research output. Domain names are coded with abbreviations. The complete coding of the domain names is available in the appendix in table \ref{table:domain-coding} as well as in the abbreviation section.

\begin{table}[htbp]
\centering
\small
\begin{tabular}{|l|cccccccccc|}
\hline
\textbf{Publishing year} & \textbf{CCH} & \textbf{EDU} & \textbf{FIN} & \textbf{HEA} & \textbf{PSB} & \textbf{SUS} & \textbf{TRA} & \textbf{SCP} & \textbf{GOV} & \textbf{MISC} \\ \hline
2011 (1) & 1 & 0 & 0 & 0 & 0 & 0 & 0 & 0 & 0 &  0 \\
2012 (1) & 0 & 0 & 0 & 0 & 0 & 0 & 0 & 0 & 0 & 1 \\
2013 (0) & 0 & 0 & 0 & 0 & 0 & 0 & 0 & 0 & 0 & 0 \\
2014 (5) & 4 & 0 & 0 & 0 & 0 & 0 & 1 & 0 & 0 & 0 \\
2015 (3) & 0 & 0 & 1 & 2 & 0 & 0 & 0 & 0 & 0 & 0 \\
2016 (7) & 3 & 0 & 0 & 1 & 1 & 1 & 0 & 1 & 0 & 0 \\
2017 (10) & 6 & 0 & 0 & 0 & 2 & 1 & 0 & 1 & 0 & 0 \\
2018 (9) & 5 & 0 & 0 & 1 & 0 & 1 & 0 & 1 & 0 & 1 \\
2019 (1) & 1 & 0 & 0 & 0 & 0 & 0 & 0 & 0 & 0 & 0 \\ \hline
\textbf{Total (37)} & 20 & 0 & 1 & 4 & 3 & 3 & 1 & 3 & 0 & 2 \\ \hline
\end{tabular}
\caption{Overall research output across domains}
\label{table:research-output}
\end{table}

Considering the number of published articles, the overall research output increased since 2011, especially in the last five years. Firstly, this can be explained by the increased adoption and knowledge of nudging. Secondly, because of the expansion of digital applications, there are also more possibilities where digital nudges can be applied.
\\

The primary research within the last ten years is conducted in the area of consumer choice. Here, a digital environment builds a typical buyer/seller relationship, where the application offers a good or service and the user is the consumer/buyer. This tendency in research shows an economic incentive behind the concept of nudging. If done right digital nudges are an excellent tool to increase conversion rates and overall revenue (\cite{mirsch_making_2018}).
However, nudging also provides benefits for others, especially complex domains, where some form of expert knowledge is necessary for decision-making. Such an example is the health domain, where five of the overall 37 research articles evaluate the usage of digital nudges. Miller et al. (\citeyear{miller_effects_2016}) study the effect of digital nudges within the MyPlate food recommendation systems. Through a feedback nudge during the pre-ordering process, they discover a significant positive effect on the meal composition of students. Those who received the MyPlate nudge while pre-ordering selected statistically significantly more fruits, vegetables, and low-fat milk than students who pre-ordered without nudging (\cite{miller_effects_2016}).
A default choice implements another example of such a digital nudge in a complex field. Because of the default choice, the amount of carbon offset payments could increase significantly. Finally, this leads to an environmentally friendly decision (\cite{szekely_nudging_2016}).

%%%%%%%%%%%%%%%%%%%%%%%

\subsection{Research type and methods}
The different research articles for this literature review use different research approaches and methods. Those articles are categorized based on Alavi's and Carlson's (\citeyear{alavi_review_1992}) research classification scheme. A graphic of this classification approach is available in the appendix (Figure \ref{fig:analysis-detail}).

\subsubsection{Non-empirical}
Non-empirical research includes articles based on the subjective opinions of the authors and literature reviews. They do not include empirically collected data (\cite{alavi_review_1992}). Overall, seven articles can be classified as non-empirical research. This accounts for approx. 19\% of the findings. Those papers follow two different non-empirical research approaches, namely literature reviews and conceptual studies. In the identified basket of literature, there is only one exception,  which creates a theoretical concept based on survey data (\cite{gamliel_average_2017}). The literature reviews present literature in the field and their findings. Conceptual studies describe theories, models or frameworks for the application of (digital) nudges. Four research articles follow both approaches. Broniarczyk and Griffin (\citeyear{broniarczyk_decision_2014}), for example, review different literature and create a model that describes which techniques can aid in the decision-making process.

\begin{table}[htbp]
\centering
\begin{tabular}{|l|ccc|}
\hline
\textbf{Non-empirical research} & \textbf{CCH} & \textbf{SCP} & \textbf{MISC} \\ \hline
Literature review (1) & 1 & 0 & 0 \\
Conceptual (2) & 1 & 1 & 0 \\
Literature review and conceptual (4) & 3 & 0 & 1 \\ \hline
\textbf{Total (7)} & 5 & 1 & 1 \\ \hline
\end{tabular}
\caption{Non-empirical research across domains}
\label{table:non-empirical}
\end{table}

As described in table \ref{table:non-empirical}, the area of consumer choice contributes the most non-empirical research. Lades (\citeyear{lades_impulsive_2014}), for example, evaluates the theoretical effect of nudges in intertemporal choices and the context of ethical usage. Thereby, the author concludes that "self-imposed nudges should be preferred to nudges by third parties" (\cite[p.122]{lades_impulsive_2014}). Furthermore, impulsive nudges should be reduced to allow more humane handling of nudges in consumer choice.

\subsubsection{Empirical}
Empirical articles are classified as articles that rely on observation and capture data through different research techniques such as surveys, case studies or laboratory experiments (\cite{alavi_review_1992}). Overall 31 articles rely on empirical methods and capture or work with some form of data.

\begin{table}[htbp]
\small
\centering
\begin{tabular}{|p{0.18\textwidth}|cccccccccc|}
\hline
\textbf{Empirical research} & \textbf{CCH} & \textbf{EDU} & \textbf{FIN} & \textbf{HEA} & \textbf{PSB} & \textbf{SUS} & \textbf{TRA} & \textbf{SCP} & \textbf{GOV} & \textbf{MISC} \\ \hline
Lab experiment (15) & 10 & 0 & 0 & 2 & 1 & 1 & 0 & 0 & 0 & 1 \\
Field experiment (5) & 2 & 0 & 0 & 1 & 1 & 1 & 0 & 0 & 0 & 0 \\
Lab experiment and field experiment (1) & 0 & 0 & 0 & 0 & 1 & 0 & 0 & 0 & 0 & 0 \\
Lab experiment and survey (3) & 2 & 0 & 0 & 0 & 0 & 0 & 0 & 1 & 0 & 0 \\
Survey (5) & 2 & 0 & 1 & 0 & 0 & 0 & 1 & 1 & 0 & 0 \\
Case Study (1) & 0 & 0 & 0 & 1 & 0 & 0 & 0 & 0 & 0 & 0 \\
Case Study, survey and lab experiment (1) & 0 & 0 & 0 & 0 & 1 & 0 & 0 & 0 & 0 & 0 \\ \hline
\textbf{Total (31)} & 16 & 0 & 1 & 4 & 4 & 2 & 1 & 2 & 0 & 1 \\ \hline
\end{tabular}
\caption{Empirical research across domains}
\label{table:empirical}
\end{table}

%Given the context of use, the location is one important aspect to keep in mind. The identified literature shows a clear focus on research in the USA and Europe. Only two studies take place in Asia. This aspect is critical to bear in mind because of different underlying mental models and mindsets. Those mindsets demand diverse requirements on the application as well as on the ethical perspective (\cite{sunstein_nudging_2015}).

\paragraph{Laboratory experiments}
In the findings of the literature, the majority (48\%) uses laboratory experiments to evaluate the efficiency and use of digital nudges. A lab experiment describes an artificial setting in which researchers can control several variables, manipulate them and evaluate the impact of that manipulation. As it can be derived from previous parts of analyses, most lab experiments take place in the field of consumer choice. Lee et al. (\citeyear{lee_monochrome_2014}) for example, study the effect of colorful versus monochrome product pictures. The finding of this study demonstrates that colorful images impact product choice in several ways and act as a kind of psychological nudge.
On the one hand, color can pull attention and highlight certain product features. On the other hand, colorful product images can create abstraction, making it harder to compare different products. This study states that marketers have to choose carefully whether to use black-and-white versus colorful imagery in advertisements and online shops.
Furthermore, lab experiments with regards to health (\cite{laran_nonconscious_2018}; \cite{langley_should_2015}), as well as sustainability (\cite{bruns_can_2018}) and pro-social behavior (\cite{zarghamee_nudging_2017}) are part of the findings in the identified literature. 
\\

Most lab experiments study the use of nudges with regards to decision information considering the underlying evaluation of the choice architecture design. Such an experiment is designed by Kretzer and Maedche (\citeyear{kretzer_designing_2018}). This study uses digital nudging in the context of enterprise recommendation agents. A precisely targeted recommendation through a social nudge allows employees to reuse existing document resources more efficiently which saves time and costs. This recommendation is a typical influence on the decision information of the choice architecture and nudges the user right at the beginning of the decision-making process.  %TODO Name sources
Five out of 15 lab experiments shape the choice architecture by changing the decision structure with regards to the choice options. In this part, the usage of specific heuristics and biases is common (\cite{tversky_judgment_1974}).
One downside of laboratory experiments is the isolated view on the decision-making process. Because of the focus on one particular research variable, this approach only evaluates the effect of a nudge in limited scope with no regards to the overall decision-making process. There is no valuation concerning the influence of the digital nudge on the whole user experience. %TODO Example?
 
\paragraph{Field experiment}
Field experiments provide a natural consideration of the application. In a field experiment, there is only limited or no control on research variables. This leads to a realistic view of the evaluation and how the user perceives a nudge. This literature review identifies five field experiments within the findings. The study by Goswami and Urminsky (\citeyear{goswami_when_2016}) combines a laboratory experiment with a field experiment while studying the effects of default effects in donations. Surprisingly, the most optimistic prediction, the significant increase of funds, is not supported. Rather, the study discovers two other effects, namely the scale-back and lower-bar effect.
\\
 
Concerning the influence on the choice architecture, field experiments grant a broad view of the whole decision-making process. Three out of a total of five field experiments observe a combination of different choice architecture elements (\cite{miller_effects_2016}; \cite{cosmo_nudging_2017}; \cite{mazar_if_2018}). Cosmo and O'Hora (\citeyear{cosmo_nudging_2017}) study the effect of time-of-use pricing models for electricity consumption in households. Thereby, a little, standalone display acts as the UI. This display gives feedback, information, and reminders about the electricity consumption of the user and therefore affects all three categories of choice architectures. The findings of the study show that informational displays cause a reduction in costs.

 \begin{table}[htbp]
\centering
\small
\begin{tabular}{|p{3.6cm}|cccc|}
\hline
\textbf{Empirical research} & \multicolumn{1}{l}{\textbf{\begin{tabular}[c]{@{}l@{}}Decision \\ Information\end{tabular}}} & \multicolumn{1}{l}{\textbf{\begin{tabular}[c]{@{}l@{}}Decision \\ structure\end{tabular}}} & \multicolumn{1}{l}{\textbf{\begin{tabular}[c]{@{}l@{}}Decision \\ assistance\end{tabular}}} & \multicolumn{1}{l|}{\textbf{Combination}} \\ \hline
Lab experiment (15) & 9 & 5 & 1 & 0 \\
Field experiment (5) & 0 & 1 & 1 & 3 \\
Lab experiment and field experiment (1) & 0 & 1 & 0 & 0 \\
Lab experiment and survey (3) & 1 & 2 & 0 & 0 \\
Survey (5) & 2 & 0 & 0 & 3 \\
Case Study (1) & 1 & 0 & 0 & 0 \\
Case Study, survey and lab experiment (1) & 1 & 0 & 0 & 0 \\ \hline
\textbf{Total (31)} & 12 & 9 & 2 & 6 \\ \hline
\end{tabular}
\caption{Empirical research across parts of the choice architecture}
\label{tabel:empirical-choice-arch}
\end{table}

\paragraph{Survey}
Surveys only account for a small part of the research. The five surveys in the findings spread across different domains. Additionally, three surveys are conducted together with a lab experiment. Those experiments take the survey data as a base and further examine the findings. One of those surveys investigates the effect of bonus-malus taxes. In combination with a social guidance nudge, users are drawn towards more sustainable public transportation options (\cite{hilton_tax_2014}). In the identified literature, surveys typically evaluate the decision information as well as a combination of choice architecture categories.

\paragraph{Case Study}
The results of the research include only one case study. Guthrie et al. (\citeyear{guthrie_nudging_2015}) study the usage of digital nudges in form of recommendations. Those recommendations should nudge people towards healthier food choices. Findings conclude that such a nudge works digitally in a better way than non-digital nudges do. Furthermore, the overall food choice is perceived as healthier, whereas the understanding of the information is still an intricate part.



%%%%%%%%%%%%%%%%%%%%%%%

\subsection{Theories and concepts used to study nudges}

%\subsubsection{Principle of Nudge} 
% Kapitel wirklich mit rein nehmen? Bläht auf und wir schon im Background Teil referenziert
% im Background chapter dann rauswerfen und hier beschreiben?
%- To fully understand the impact of the nudge their primary principle and goal is essential to understand
%- In 33 of the overall 37 research papers on of the 6 digital nudge principles by \cite{weinmann_digital_2016} which are based on \cite{thaler_nudge:_2009} can be identified.
%- Single use or combination of different principles
%\paragraph{Incentive}
%-The majority emphasized incentive behind a digital nudge
%- Such an example can be found in ... 
%\paragraph{Understanding mapping}
%- 3 papers used nudges that supports the understanding of mapping
%- helps in complex environments
%- One example of that is product comparison... 
%\paragraph{Defaults}
%- Very powerful option in offline scenarios \cite{johnson_defaults_2003}
%- Also digital environments use defaults very efficient
%- Such as in charitable giving
%\paragraph{Giving Feedback}
%- Efficient tool for decision making in complex choices and domains
%- Example in the health sector
%- Giving feedback on meal composition in schools
%\paragraph{Expecting Error}
%- Guiding choice also involves errors
%- Some nudges can guide the user to better decisions if they calculate some errors into the process
%- In the identified papers, two are using a nudge that expects an error
%\paragraph{Structure complex choices}
%- One of the primary goals of nudges is also the structuring of complex choices and complex choice types
%- This can also include intertemporal choices such as in ...
%- all in all eight nudges try to structure complex choices in some way
%- This kind of nudge principle is often found complex domains or in complex product choices where some form of expert knowledge is needed

\subsubsection{Conceptual Background} 
It is difficult to evaluate the overall conceptual background of the identified literature. Digital nudging is a concept that is based on research across several research streams and domains. The same accounts for the underlying concepts, theories, and models. Within the findings, 29 research articles mention one or more theoretical background concepts they refer to during their studies. All in all, those papers mention 25 different theories. From those 25 theories, the majority is only named once. This distribution of theories shows the multidisciplinary origin of digital nudging.
\\

The most mentioned theories are the prospect theory with eight mentions (\cite{kahneman_prospect_1979}). The libertarian paternalism (\cite{thaler_nudge:_2009}) with seven mentions, the model of judgment under uncertainty with six mentions (\cite{tversky_judgment_1974}) and the bounded rationality, an important basic theory for decision making, with five mentions (\cite{simon_behavioral_1955}) . All those theories and models have their origin in behavioral economic research and focus on decision making.
For social nudges, the theory of social influence (\cite{cialdini_social_2004}) is named three times in the findings. The social influence theory "emphasizes the way in which [...] goals interact with external forces" (\cite[p.591]{cialdini_social_2004}). This kind of influence is subtle, indirect and outside of awareness.
Research articles that lay a focus on psychological factors, reference the reactance theory (\cite{brehm_theory_1966}), the general evaluability theory (\cite{hsee_general_2010}) and the construal level theory (\cite{trope_construal-level_2010}). Every single theory is mentioned two times. Brehm (\citeyear{brehm_theory_1966}) states that the theory of psychological reactance shows that individuals have certain freedoms concerning their behavior. If these behavioral freedoms are reduced or threatened, the individual will be motivated to regain them. This theory provides an important insight into how far nudges should take influence in decision-making and what their boundaries are. The general evaluability theory, on the other hand, focuses on the value system of individuals. It specifies "when people are value sensitive and when people mispredict their own or others' value sensitivity" (\cite[p.343]{hsee_general_2010}). Those insights have important meanings for the design of social nudges. The construal level theory emphasizes cognitive and mental processes with regards to similarity and comparisons (\cite{trope_construal-level_2010}).
Furthermore, the majority of research articles in the domain of health, mentions the health belief model (\cite{rosenstock_health_1974}). This model evaluates three categories of preventive health behavior. With a focus on mental states, it describes a model with "states which help to account for behavior" (\cite[p.354]{rosenstock_health_1974}) in the domain of health. Such insights in behavioral psychology provide essential guidelines for the efficient design of health nudges. 


\subsubsection{Heuristics and biases}
Heuristics act as a rule of thumb for guiding a choice in cognitive loaded environments (\cite{thaler_nudge:_2009}). In the 37 identified papers, 21 articles reference or use a heuristic in the design and implementation of a digital nudge. Zarghamee et al. (\citeyear{zarghamee_nudging_2017}) execute the only study that relates to the status quo bias as well as social and moral norms. In this study, two implemented nudges increase the donation to charitable giving about 25\%. This gain is achieved with a set default and an additional social nudge that provides a social reference point.
The most used form of heuristics are norms, which describes the effect that people tend to be influenced by the behavior of others (\cite{schneider_digital_2018}). Choice architects can implement nudges that use norms in two ways. Where social norms nudge users towards a social reference point (\cite{wang_socially_2018}), moral norms tend to emphasize value-based decisions. In a field experiment of 2017 a "pay what you want" pricing model was changed to nudge users to a higher pricing decision at a book store. The result shows that members, who are reminded of their club membership right before the decision, significantly adjust their pricing decision upwards (\cite{gravert_pride_2017}).

\begin{table}[htbp]
\centering
\begin{tabular}{|l|cccc|}
\hline
\textbf{Heuristic / Bias} & \multicolumn{1}{l}{\textbf{\begin{tabular}[c]{@{}l@{}}Decision \\ information\end{tabular}}} & \multicolumn{1}{l}{\textbf{\begin{tabular}[c]{@{}l@{}}Decision \\ structure\end{tabular}}} & \multicolumn{1}{l}{\textbf{\begin{tabular}[c]{@{}l@{}}Decisions \\ assistance\end{tabular}}} & \multicolumn{1}{l|}{\textbf{Combination}} \\ \hline
Status quo bias (5) & 0 & 5 & 0 & 0 \\
Decoy effect (1) & 0 & 1 & 0 & 0 \\
Primacy and recency effect (1) & 0 & 1 & 0 & 0 \\
Middle-option bias (0) & 0 & 0 & 0 & 0 \\
Anchoring and adjustments (1) & 1 & 0 & 0 & 0 \\
Norms (12) & 8 & 1 & 2 & 1 \\
Status quo bias and norms (1) & 0 & 1 & 0 & 0 \\
Scarcity effect (0) & 0 & 0 & 0 & 0 \\ \hline
\textbf{Total (21)} & 9 & 9 & 2 & 1 \\ \hline
\end{tabular}
\caption{Heuristics used across parts of choice architectures}
\label{table:heuristics-choice}
\end{table}

Five of the 21 research articles that use heuristics and biases in a certain way referred to the status quo bias, which is a form of default. Defaults are one of the most efficient forms of nudging (\cite{johnson_defaults_2003}). The status quo bias shows that people tend to "favor the status quo, so they are less inclined to change default options" (\cite[p.71]{schneider_digital_2018}). One interesting finding of the identified literature is shown by Steffel et al. (\citeyear{steffel_ethically_2016}). In this study, the researchers try to de-bias the effect of a default choice by communicating the nudge transparently. The result shows that it is not entirely possible to de-bias a default choice and that the user is still nudged towards the default value. This result shows the cognitive strength of the status quo bias.
\\

Other heuristics and biases are used less. Such an example is the decoy effect. By showing an unattractive option besides an attractive one, the user is nudged towards the attractive option (\cite{schneider_digital_2018}). Only one research article studies this effect. The target environment is an online crowdsourcing platform where users can donate money to an upcoming project. Here, the user has the choice between different donation-packages that include some reward. With the help of the decoy effect, the study shows that donations can increase by 11\% (\cite{tietz_decoy_2016}).
Another less used heuristic and bias is the primacy and recency effect which describes the effect of the positioning of choice options and interface elements. One research article studies the effect of choice positioning across different sides of the screen. For food choices, the researchers find out that it is more likely to choose a healthy option when it is located on the left side of the screen. Overall, several visual cues play an important role in influencing choices, such as the positioning and ordering of choice options. (\cite{romero_healthy-left_2016}).
\\

Surprisingly, two heuristics often implemented in non-digital nudges are not present in the findings of the identified literature. First, this includes the middle-option bias, which explains that people confronted with three options are most likely to choose the middle-option considering the price or size of a product (\cite{schneider_digital_2018}). In an offline environment, one can observe this effect in several coffee-shops. Here, the customer typically has a choice between three options in size. Because of the middle-option bias, it is most likely that the customer will choose the medium-sized coffee. Such a bias could also be transferred to digital environments, for example in a scenario where the user has to choose between different pricing models or product configurations. 
Furthermore, none of the research articles studies the implementation of a nudge that is based on the scarcity effect. This effect shows that people tend to perceive rare items as more attractive and desirable (\cite{gergen_search_1980}). One crucial aspect of the scarcity effect is a difficult implementation. Users often perceive nudges that build on top of this effect unethical (\cite{sunstein_nudging_2015}), while at the same time they are useful in forcing certain choices. An example of this nudge is described in the introduction chapter.
\\

By mapping those heuristics to categories of the choice architecture, several insights can be made. The first observation is that norms are typically used for decision information. In eight out of twelve cases, norms are used in nudges that provide decision information. At the same time, norms are also efficient when it comes to decision assistance. Such a social reference point provides orientation for users and makes choices more accessible. 
Other heuristics are more efficient in decision-making when it comes to the decision structure itself. In every study the decision structure is changed, researchers use the status quo bias.
Other techniques like the decoy effect and the primacy and recency effect are also used to manipulate decision structure.


%%%%%%%%%%%%%%%%%%%%%%%

\subsection{Influence on the choice architecture and decision making}

%\subsubsection{Type of choice}
%\paragraph{Binary}
%\paragraph{Discrete choice}
%\paragraph{Continuous}
%\paragraph{Any type of choice \& inter temporal}

Another vital aspect shows the design of the choice architecture. Different parts of the choice architecture should be altered according to the part of the decision-making process to guide choices efficiently. For example, it is not perceived as useful to design a nudge for decision information, when the user is right in the decision process. Such nudges should always be implemented beforehand. Since those decision steps have different requirements, the use of the right heuristics and biases is essential. Table \ref{table:heuristics-choice} shows the heuristics and biases used to guide decisions in different categories of the choice architecture. Here, the status quo bias is solely implemented in nudges which influence the decision structure. The same holds for heuristics and biases concerning the decision information. Besides the heuristics, several other possibilities to influence the choice architecture exist. This can be plain informational text and the visibility of information (used by twelve research articles). Also, the translation of information (used by eight research articles) is a simple decision that designers can make to guide choices. 

\begin{table}[htbp]
\centering
\small
\begin{tabular}{|p{0.15\textwidth}|cccccccccc|}
\hline
\textbf{Choice Architecture} & \textbf{CCH} & \textbf{EDU} & \textbf{FIN} & \textbf{HEA} & \textbf{PSB} & \textbf{SUS} & \textbf{TRA} & \textbf{SCP} & \textbf{GOV} & \textbf{MISC} \\ \hline
Decision information (15) & 10 & 0 & 1 & 2 & 1 & 0 & 0 & 0 & 0 & 1 \\
Decision structure (10) & 4 & 0 & 0 & 0 & 2 & 2 & 0 & 2 & 0 & 0 \\
Decision assistance (3) & 2 & 0 & 0 & 1 & 0 & 0 & 0 & 0 & 0 & 0 \\
Combination (6) & 2 & 0 & 0 & 1 & 0 & 1 & 1 & 1 & 0 & 0 \\ \hline
\textbf{Total (34)} & 18 & 0 & 1 & 4 & 3 & 3 & 1 & 3 & 0 & 1 \\ \hline
\end{tabular}
\caption{Choice architecture parts used across the domains}
\label{tabel:choice-arch-domains}
\end{table}

Across the domains, there is no real pattern to be recognized. The vast majority of the research articles explores the impact of choice information on consumer choice. In this decision environment consumers typically decide on their own, with no need for expert knowledge. Many consumers nowadays use technology to compare different choices and products. Accurate decision information is a powerful tool to guide those choices. 
However, the simplification of decision information proves to be beneficial in other domains, too, for example in the health domain. Langley et al. (\citeyear{langley_should_2015}) study the effect of decision information in an online forum to nudge users towards a vaccination decision. The finding shows that the decision information provides helpful advice for users, but at the same time demonstrates that vaccination decisions are not taken in social isolation. This experiment implements a complex nudge, by targeting the user in a digital environment, whereas the decision has to be made in a non-digital environment. For such specific nudges, no further literature to date is identified.
In terms of decision structure, heuristics such as the status quo bias, the primacy and recency effect, and the decoy effect have proved to be effective. With six research articles, the majority manipulates choice defaults to nudge users towards certain decisions. As mentioned before, defaults provide a valid form of nudging. Other techniques used to design effective nudges are about the choice options, their position, effort and also consequences (\cite{munscher_review_2016}). One of the studies explores the framing effects of intertemporal choices (\cite{faralla_framing_2017}). Here the authors try to nudge users towards a more future-oriented decision. Participants of the study have the option between two amounts of money such as "55€ today or 75€ in 60 days" (\cite[p.13]{faralla_framing_2017}). The experiment additionally gives information about an explicit penalty if participants take the money now. This gentle nudge leads towards a more future-oriented decision by changing only the consequences of the choice. Concerning domains, no real pattern can be recognized for the category of decision structure.
\\

A minor part of the research articles studies nudges with regards to decision assistance. Decision assistance can be achieved by providing reminders or facilitate commitment for a choice (\cite{munscher_review_2016}). In a study of 2018 customers were nudged with the help of planning prompts. Those prompts asked for a specified timeframe for paying credit card debts. Within the help of this particular nudge, the likelihood for following the set intentions increases significantly (\cite{mazar_if_2018}).
Research articles that emphasize a longer decision-making process or intertemporal decisions often use nudges in the category of decision assistance.
\\

Another crucial finding is that only six research articles use a combination of choice architecture techniques. An effective nudging can only happen if the entire decision-making process is taken into scope of the nudge (\cite{miller_effects_2016}; \cite{hilton_tax_2014}; \cite{cosmo_nudging_2017}; \cite{mazar_if_2018}; \cite{basu_choosing_2017}; \cite{schneider_nudging_2017}). 

%- Here, future research can close a gap with field experiments and the effects of multiple nudges on the decision-making process