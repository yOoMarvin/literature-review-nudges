\section{Results}

\subsection{Overall research output}
Since the release of the "Nudge" by Thaler and Sunstein in 2009, the concept of nudges gained more and more interest in several research streams and domains. Table \ref{table:research-output} gives an overview of the overall research output. To get a better overview, the domain names are coded with abbreviations. The complete coding of the domain names is available in the appendix in table \ref{table:domain-coding} as well as in the abbreviation section.

Considering the amount published articles, the overall research output increased since 2011. Especially in the last five years the research has gained interest. One the one hand this can be explained by the increased adoption and knowledge of nudging. On the other hand there are also more possibilities where digital nudges can be used. More and more domains face the digitization. This leads to an expansion of digital applications and their adoption. %TODO find quote
Digital nudges also gained awareness in across several domains. Thereby, the main research within the last ten years is done in the area of consumer choice. Here a digital environment builds a typical buyer / seller relationship, where the application offers some kind of good or service and the user takes the role of the consumer / buyer. One explanation for that is the economic interest behind the concept of nudging. If done right digital nudges are a good tool to increase conversion rates and overall revenue. %TODO Source finden
But the interest in digital nudging also spans across other domains. Especially complex domain, where typically some form of expert knowledge is needed in decision-making. Such an example is the health domain, where five of the overall 37 research articles evaluate the usage of digital nudges. \cite{miller_effects_2016} study the effect of digital nudges within the MyPlate food recommendation systems. Through a feedback nudge during the pre-ordering process they discover a significant positive effect on the meal composition of students. "Students who received the MyPlate nudge while pre-ordering selected statistically significantly more fruits, vegetables, and low-fat milk than students who pre-ordered without nudging" \cite{miller_effects_2016}.
Another example of such a digital nudge in a complex field describes \cite{szekely_nudging_2016}. With a nudge in a default choice the amount of carbon offset payments could be increased significantly. Finally this leads to an environment friendly decision.


\begin{table}[htbp]
\centering
\small
\begin{tabular}{l|cccccccccc}
\textbf{Publishing year} & \textbf{CCH} & \textbf{EDU} & \textbf{FIN} & \textbf{HEA} & \textbf{PSB} & \textbf{SUS} & \textbf{TRA} & \textbf{SCP} & \textbf{GOV} & \textbf{MISC} \\ \hline
2011 (1) & 1 & 0 & 0 & 0 & 0 & 0 & 0 & 0 & 0 &  0 \\
2012 (1) & 0 & 0 & 0 & 0 & 0 & 0 & 0 & 0 & 0 & 1 \\
2013 (0) & 0 & 0 & 0 & 0 & 0 & 0 & 0 & 0 & 0 & 0 \\
2014 (5) & 4 & 0 & 0 & 0 & 0 & 0 & 1 & 0 & 0 & 0 \\
2015 (3) & 0 & 0 & 1 & 2 & 0 & 0 & 0 & 0 & 0 & 0 \\
2016 (7) & 3 & 0 & 0 & 1 & 1 & 1 & 0 & 1 & 0 & 0 \\
2017 (10) & 6 & 0 & 0 & 0 & 2 & 1 & 0 & 1 & 0 & 0 \\
2018 (9) & 5 & 0 & 0 & 1 & 0 & 1 & 0 & 1 & 0 & 1 \\
2019 (1) & 1 & 0 & 0 & 0 & 0 & 0 & 0 & 0 & 0 & 0 \\ \hline
\textbf{Total (37)} & 20 & 0 & 1 & 4 & 3 & 3 & 1 & 3 & 0 & 2
\end{tabular}
\caption{Overall research output across domains}
\label{table:research-output}
\end{table}

%%%%%%%%%%%%%%%%%%%%%%%

\subsection{Research type and methods}
While main focus of this paper is on nudges in consumer choice digital nudges are also used and researched in other domains and fields of application. The different research articles, that were identified for this literature review use different research approaches and methods. Those articles shoould be categorized based on the Alavi and Carlson's research classification scheme (\cite{alavi_review_1992}). A graphic of this classification approach is available in the appendix (\ref{fig:analysis-detail}).

\subsubsection{Non-empirical}
\begin{table}[htbp]
\centering
\begin{tabular}{l|ccc}
\textbf{Non-empirical research} & \textbf{CCH} & \textbf{SCP} & \textbf{MISC} \\ \hline
Literature review (1) & 1 & 0 & 0 \\
Conceptual (2) & 1 & 1 & 0 \\
Literature review and conceptual (4) & 3 & 0 & 1 \\ \hline
\textbf{Total (7)} & 5 & 1 & 1
\end{tabular}
\caption{Non-empirical research across domains}
\label{table:non-empirical}
\end{table}

Non-empirical research includes articles based on the subjective opinions of the authors and/or literature reviews. They do not include empirically collected data \cite{alavi_review_1992}. In the identified basket of literature there is only one exception, where \cite{gamliel_average_2017} which creates theoretical concept based on a survey. Overall, seven articles can be classified as non-empirical research. This makes ca. 19\% of the findings. Those papers use in particular two different, non-empirical research approaches. Literature reviews and conceptual studies. The literature reviews are present literature in the field and their findings, where as conceptual studies describe theories, models or frameworks for the application of (digital) nudges. Four research articles do follow both approaches. \cite{broniarczyk_decision_2014} for example reviews different literature and creates a model that describes which techniques can aid in the decision-making process.

As described in  \ref{table:non-empirical} most non-empirical research is done in the area of consumer choice. \cite{lades_impulsive_2014} for example evaluates the theoretical effect of nudges in inter temporal choices and the context of ethical usage. Thereby, he concludes that "self-imposed nudges should be preferred to nudges by third parties". Furthermore, impulsive nudges should be reduced to allow more ethical handling of nudges in consumer choice.


\subsubsection{Empirical}
\begin{table}[htbp]
\small
\centering
\begin{tabular}{p{3.6cm}|cccccccccc}
\textbf{Empirical research} & \textbf{CCH} & \textbf{EDU} & \textbf{FIN} & \textbf{HEA} & \textbf{PSB} & \textbf{SUS} & \textbf{TRA} & \textbf{SCP} & \textbf{GOV} & \textbf{MISC} \\ \hline
Lab experiment (15) & 10 & 0 & 0 & 2 & 1 & 1 & 0 & 0 & 0 & 1 \\
Field experiment (5) & 2 & 0 & 0 & 1 & 1 & 1 & 0 & 0 & 0 & 0 \\
Lab experiment and field experiment (1) & 0 & 0 & 0 & 0 & 1 & 0 & 0 & 0 & 0 & 0 \\
Lab experiment and survey (3) & 2 & 0 & 0 & 0 & 0 & 0 & 0 & 1 & 0 & 0 \\
Survey (5) & 2 & 0 & 1 & 0 & 0 & 0 & 1 & 1 & 0 & 0 \\
Case Study (1) & 0 & 0 & 0 & 1 & 0 & 0 & 0 & 0 & 0 & 0 \\
Case Study, survey and lab experiment (1) & 0 & 0 & 0 & 0 & 1 & 0 & 0 & 0 & 0 & 0 \\ \hline
\textbf{Total (31)} & 16 & 0 & 1 & 4 & 4 & 2 & 1 & 2 & 0 & 1
\end{tabular}
\caption{Empirical research across domains}
\label{table:empirical}
\end{table}


\begin{table}[htbp]
\centering
\small
\begin{tabular}{p{3.6cm}|cccc}
\textbf{Empirical research} & \multicolumn{1}{l}{\textbf{\begin{tabular}[c]{@{}l@{}}Decision \\ Information\end{tabular}}} & \multicolumn{1}{l}{\textbf{\begin{tabular}[c]{@{}l@{}}Decision \\ structure\end{tabular}}} & \multicolumn{1}{l}{\textbf{\begin{tabular}[c]{@{}l@{}}Decision \\ assistance\end{tabular}}} & \multicolumn{1}{l}{\textbf{Combination}} \\ \hline
Lab experiment (15) & 9 & 5 & 1 & 0 \\
Field experiment (5) & 0 & 1 & 1 & 3 \\
Lab experiment and field experiment (1) & 0 & 1 & 0 & 0 \\
Lab experiment and survey (3) & 1 & 2 & 0 & 0 \\
Survey (5) & 2 & 0 & 0 & 3 \\
Case Study (1) & 1 & 0 & 0 & 0 \\
Case Study, survey and lab experiment (1) & 1 & 0 & 0 & 0 \\ \hline
\textbf{Total (31)} & 12 & 9 & 2 & 6
\end{tabular}
\caption{Empirical research across parts of the choice architecture}
\label{tabel:empirical-choice-arch}
\end{table}

Empirical articles are classified as articles that rely on observation and capture data through different research techniques such as survey, case studies or laboratory experiments \cite{alavi_review_1992}. Overall 31 articles emphasize empirical methods and capture or work with some form of data.

Given the context of use, the location is one important aspect to keep in mind. The identified literature shows a clear focus on research in the USA and Europe. Only two studies take place in Asia. This aspect is important to bear in mind because of different underlying mental models and mindsets. Those mindsets demand diverse requirements on the application as well as on the ethical perspective \cite{sunstein_nudging_2015}.

\paragraph{Laboratory experiments}
In the findings of literature the majority (48\%) uses laboratory experiments to evaluate the efficiency and use of digital nudges. A lab experiment describes an artificial setting in which researches can control several variables, manipulate them and evaluate the impact of that manipulation. This kind of research is ideally as a research method for digital nudging. As can be seen from previous parts of analysis, most lab experiments take place in the field of consumer choice. For example, \cite{lee_monochrome_2014} study the effect of colorful versus monochrome product pictures. The finding of this study is that colorful images impact the product choice in several ways and act as a kind of "psychological nudge". On the one hand color can pull attention and highlight certain product features. On the other hand colorful product images can create some kind of abstraction that makes in harder to compare different products. \cite{lee_monochrome_2014} state that markets have to choose carefully whether to use black-and-white versus colorful imagery in advertisements and online shops.
Furthermore, lab experiments with regards to health (\cite{laran_nonconscious_2018}; \cite{langley_should_2015}), as well as sustainability (\cite{bruns_can_2018}) and pro-social behavior (\cite{zarghamee_nudging_2017}) are part of the findings. Focusing on the underlying evaluation of the choice architecture design most lab experiments study the use of nudges concerning decision information, which typically takes place as the first step right before the decision. Such an experiment is designed by \cite{kretzer_designing_2018}. In this study digital nudging is used in the context of enterprise recommendation agents. A precisely targeted recommendation through a social nudge allows employees to reuse existing document resources more effectively which saves time and costs. This recommendation is a typical influence on the decision information of the choice architecture and nudges the user right in the beginning of the decision-making process. 
Another vast part of lab experiments shapes the decision structure of the choice architecture. Here, the choice architect manipulates the decision itself, often through the change of choice options. In this part the usage if certain heuristics and biases is common \cite{tversky_judgment_1974}.
One downside of laboratory experiments is the isolated view on the decision-making process. Because of the artificial setting and variables set in advance, those studies only evaluate the effect of a nudge on only one part of the overall decision-making. There is no measurement concerning the digital nudge influence on the whole process.
 
\paragraph{Field experiment}
Field experiments provide exactly this natural consideration of the application. In a field experiment there is only limited or no control on research variables. This leads to a realistic view on the evaluation and how a nudge is perceived by the user. This literature review identifies five field experiments within the findings. The research article of \cite{goswami_when_2016} combines a laboratory experiment with a field experiment while studying the effects of default effects in donations. Surprisingly, the most optimistic prediction , the significant increase of funds, is not supported. Rather, two other effects can be discovered. The "scale-back" and "lower-bar" effect. 
Concerning the influence on the choice architecture, field experiments grant a broad view on the whole decision making process. Three out of the five field experiments take a look at a combination of choice architecture elements (\cite{miller_effects_2016}; \cite{cosmo_nudging_2017}; \cite{mazar_if_2018}). \cite{cosmo_nudging_2017} study the effect of time-of-use pricing models for electricity consumption in households. Thereby, a little, standalone display acts as the UI. This display gives feedback, information and reminders about electricity consumption of the user. This is a digital nudge that effects all three categories of the choice architecture. The finding of the study shows that informational displays cause a reduction of costs.

\paragraph{Survey}
Survey make only a small part of the research in the topic of digital nudges. This can be explained because of the experimental nature of nudges and survey data is not current. The five surveys in the findings spread across different domains. Addiontally, three survey are done together with a lab experiment. Those experiments take the survey data as a base and further examine the findings. The survey of \cite{hilton_tax_2014} investigates the effect of bonus malus taxes. In combination with a social guidance nudge, users are drawn towards more sustainable transportation options. Surveys typically evaluate the decision information as well as a combination of choice architecture categories.

\paragraph{Case Study}
Within the results of the research, only one case study is identified. \cite{guthrie_nudging_2015} the usage of digital nudges in the form of recommendations. Those recommendations should nudge people towards healthier food choices. Findings conclude that such a nudge works digital in a better way than non-digital nudges do. Furthermore, the overall food choice is perceived as healthier, where as the understanding of the information still is a difficult part.


%%%%%%%%%%%%%%%%%%%%%%%

\subsection{Theories and concepts used to study nudges}

\subsubsection{Principle of Nudge} %TODO
%TODO im Background chapter dann rauswerfen und hier beschreiben?
%- To fully understand the impact of the nudge their primary principle and goal is essential to understand
%- In 33 of the overall 37 research papers on of the 6 digital nudge principles by \cite{weinmann_digital_2016} which are based on \cite{thaler_nudge:_2009} can be identified.
%- Single use or combination of different principles
%\paragraph{Incentive}
%-The majority emphasized incentive behind a digital nudge
%- Such an example can be found in ... 
%\paragraph{Understanding mapping}
%- 3 papers used nudges that supports the understanding of mapping
%- helps in complex environments
%- One example for that is product comparison.... 
%\paragraph{Defaults}
%- Very powerful option in offline scenarios \cite{johnson_defaults_2003}
%- Also digital environments use defaults very efficient
%- Such as in charitable givings
%\paragraph{Giving Feedback}
%- Efficient tool for decision making in complex choices and domains
%- Example in the health sector
%- Giving feedback on meal composition in schools
%\paragraph{Expecting Error}
%- Guiding choice also involves errors
%- Some nudges can guide the user to better decisions if they calculate some kind of errors into th eprocess
%- In the identified papers, 2 are using a nudge that expects error
%\paragraph{Structure complex choices}
%- One of the primary goals of nudges is also the structuring of complex choices and complex choice types
%- This can also include inter temporal choices such as in ...
%- all in all 8 nudges try to structure complex choices in some way
%- This kind of nudge principle is often found complex domains or in complex product choices where some form of expert knowledge is needed

\subsubsection{Conceptual Background} %TODO --> Also add numbers when mentioning
- Overall difficult to evaluate
- digital nudging is a concept from across several research streams
- That's why underlying concepts and background theories and models are also spread across different domains
- Overall 29 papers mentioned one or more theoretical background concepts
- 25 different theories mentioned
- A lot only named one and therefore not rather relevant
- Most mentioned theories are the libertarian paternalism, bounded rationality, judgment under uncertainty and prospect theory
- All build the theoretical background for nudging and digital nudging
- Come from behavioral economics
- For social nudges social influence theory
- Paper that laid a focus on psychology reference psychological reactance theory and the general evaluabilty theory, construal level theory
- Papers that use nudges in the domain of health mention the health belief model


\subsubsection{Heuristics and biases}
\begin{table}[htbp]
\centering
\begin{tabular}{l|cccc}
\textbf{Heuristic / Bias} & \multicolumn{1}{l}{\textbf{\begin{tabular}[c]{@{}l@{}}Decision \\ information\end{tabular}}} & \multicolumn{1}{l}{\textbf{\begin{tabular}[c]{@{}l@{}}Decision \\ structure\end{tabular}}} & \multicolumn{1}{l}{\textbf{\begin{tabular}[c]{@{}l@{}}Decisions \\ assistance\end{tabular}}} & \multicolumn{1}{l}{\textbf{Combination}} \\ \hline
Status quo bias (5) & 0 & 5 & 0 & 0 \\
Decoy effect (1) & 0 & 1 & 0 & 0 \\
Primacy and recency effect (1) & 0 & 1 & 0 & 0 \\
Middle-option bias (0) & 0 & 0 & 0 & 0 \\
Anchoring and adjustments (1) & 1 & 0 & 0 & 0 \\
Norms (12) & 8 & 1 & 2 & 1 \\
Status quo bias and norms (1) & 0 & 1 & 0 & 0 \\
Scarcity effect (0) & 0 & 0 & 0 & 0 \\ \hline
\textbf{Total (21)} & 9 & 9 & 2 & 1
\end{tabular}
\caption{Heuristics used across parts of choice architectures}
\label{table:heuristics-choice}
\end{table}

%TODO
- To make a nudge succesful, some kind of heuristic (rule of thumb is used) \cite{thaler_nudge:_2009}
- In the 37 papers, 21 used heuristic in the evaluation of a digital nudge 
- Paper of ... used combination of norms and status quo bias
- Most used are norms - mental model, psychological focus --> Schneider Paper ?!
- Charitable givings, social influence, social defaults
- 5 papers used status quo bias which are some form of default
- Other biases where less used
- Decoy effect, which actually has a promising influence on online choice architecture
- Often used in crowd funding campaigns
- In example paper ...
- Primacy and recency effect, used the positioning
- Such as in healthy left, healthy right
- 2 heuristics often used in non-digital nudges
- Middle option bias - Starbucks, people tend to choose the middle option
- Scarcity effect - from example in introduction
- Often an unethical use \cite{sunstein_nudging_2015} but very effective in forcing choices
- Mapping those heuristics to the parts of the choice architecture several insights can be made
- Norms typically used for information
- In 8 of the 12 cases norms were used in nudging they provided decision information
- At the same time, norms are also efficient when it comes to long term decision and decision assistance
- Social reference point makes choice easier
- Other heuristics are more efficient in the decision making when it comes to choice structure
- Choice itself
- All of the 5 times were the status quo bias is exploited it was decision structure (SOURCES)
- Other techniques like the decoy effect and the primacy and recency effect were also used in decision structure
- Those techniques influence decision directly when the decision is taking
- Such in .... (example)

%%%%%%%%%%%%%%%%%%%%%%%

\subsection{Influence on the choice architecture and decision making}

%\subsubsection{Type of choice}
%\paragraph{Binary}
%\paragraph{Discrete choice}
%\paragraph{Continuous}
%\paragraph{Any type of choice \& inter temporal}

\begin{table}[htbp]
\centering
\small
\begin{tabular}{p{3.5cm}|cccccccccc}
\textbf{Choice Architecture} & \textbf{CCH} & \textbf{EDU} & \textbf{FIN} & \textbf{HEA} & \textbf{PSB} & \textbf{SUS} & \textbf{TRA} & \textbf{SCP} & \textbf{GOV} & \textbf{MISC} \\ \hline
Decision information (15) & 10 & 0 & 1 & 2 & 1 & 0 & 0 & 0 & 0 & 1 \\
Decision structure (10) & 4 & 0 & 0 & 0 & 2 & 2 & 0 & 2 & 0 & 0 \\
Decision assistance (3) & 2 & 0 & 0 & 1 & 0 & 0 & 0 & 0 & 0 & 0 \\
Combination (6) & 2 & 0 & 0 & 1 & 0 & 1 & 1 & 1 & 0 & 0 \\ \hline
\textbf{Total (34)} & 18 & 0 & 1 & 4 & 3 & 3 & 1 & 3 & 0 & 1
\end{tabular}
\caption{Choice architecture parts used across the domains}
\label{tabel:choice-arch-domains}
\end{table}

%TODO 
- Another important aspect is the influence of the nudge design on the choice architecture and the overall part of decision-making
- \cite{munscher_review_2016} classified this parts of decision making in a taxonomy consisting out of decision information, structure and assistance
- To guide choices efficient, different parts of the choice architecture should be designed different and according to the decision-making process
- For example it does not make any sense to design a nudge for decision information, when the user is directly in the decision process
- Such nudges should always be designed beforehand
- important aspect: right use of heuristics and biases very important. See table
- Special biases used to guide choice in decision structure
- Same goes for biases in decision information
- Norms very efficient
- Outside the heuristics there are several other possibilities to influence the choice architecture
- plain informational text and making information visible (12)
- Also the translation (8) is a simple design decision tat designer can make to guide choices and influence right from the start with regards to choice information
-  Across the domains there is no real pattern to be recognized
- Most part of the papers explored impact of choice information in consumer choice
- In this area the consumer decides on his or her own, now expert knowledge
- Lot of people use technology to compare different choices and products
- Decision information is powerful tool to guide those choices
- But also beneficial in other domains e.g. health --> Example!
- In terms of decision structure heuristics such as the status quo effect, primacy and recency effect and decoy effect have proven to be efficient
- As well as for decision information, also other techniques provide good results
- Most popular is the change of choice defaults (6)
- Very effective. That's why defaults are a complete nudge principles for themselves
- Other techniques used to design effective nudges are about the options. About their positions, effort and also consequences
- manipulating these parameters in a choice architecture can nudge people towards certain decisions
- Such as in ....
- For domains, no real pattern can be recognized
- Most used in choice architecture, most possibilities
- A not so big part of the papers evalutes decisions assistance
- This can be done by providing reminders or facilitate commitment for a choice
- Examples!
- Decisions assistance is often used in domains where longer choices are made
- Such as health and sustainability
- One crucial finding is that only 6 papers used a combination of choice architecture techniques
- This is due to experiment design in laboratory setting
- But also isolated view on nudge
- Effective nudging can only happen if whole decision making process is viewed and designed
- Papers like ... 
- Here, future research can close a gap with field experiments and the effective of multiple nudges on the decision making process