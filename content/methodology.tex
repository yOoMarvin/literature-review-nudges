\section{Methodology}
This literature review follows a systematic approach, that is well-established in the discipline of information systems (\cite{webster_analyzing_2002}) and targets a qualitative subset of literature. Thereby trying to be as representative as possible. The focus lies on publications of academic journals. To get a qualitative representation of the current research, a journal-wise analysis is preferable to a database-based analysis. The overall approach follows a known pattern in information systems literature reviews (\cite{alavi_review_1992}).
\begin{enumerate}
\item Identifying, reviewing and analyzing existing literature in the field of digital nudges. This includes empirical, as well as non-empirical studies.
\item Identifying theoretical and methodological approaches used to understand the use of nudges in consumer choice. This also includes used heuristics and the design of a choice architecture.
\item Identifying a research gap within existing literature to guide future research.
\end{enumerate}

To realize this strategy, several variables are necessary for the search process. Digital nudging is a concept that spans across several fields of research. At the same time the understanding as well as the implementation differs widely. Therefore, this literature review aims to cooperate  different research streams to build a common ground, by identifying and analyzing the most representative research articles in the domain. Search variables are set journal- and paper-wise. A graphic of the screening process is available in the appendix (Figure \ref{fig:method}).

\subsection{Journal selection}
Journal-wise variables are the journal domain and its rating. As suggested in existing literature, it is reasonable to not only search within the field of information systems but also in other research streams. (\cite{webster_analyzing_2002}). It is reasonable to examine academic journals with the most influence in the research domain. This includes research from the area of information systems, management, marketing, behavioral economics, and psychology. Regarding research about information systems the \textit{AIS Basket of 8} provides a good starting point (\cite{alavi_review_1992}). This scholarly basket consists out of eight well-respected journals in the domain. After the AIS scholarly basket, academic journals about management and marketing are recorded in the research process. Thereby, the journal list of the \textit{UT Dallas} is taken as a reference point. Overall, the journal list of the UT Dallas contributes with twelve journals to the research pool. 
Relevant publications from the domains of behavioral economics, decision making and psychology are identified by the \textit{VHB} journal rating \textit{JOURQUAL3}\footnote{more information under \url{https://vhbonline.org/vhb4you/jourqual/vhb-jourqual-3/gesamtliste/}}. Journals with a rating of \textit{B} or better are taken into account for the review.
To finalize the list of sources for the upcoming analysis, conference publication from the AIS with a VHB rating of \textit{B} or better are included, too. In total 32 academic journals were examined. A complete list of these journals is accessible in the appendix (Table \ref{table:journals}).

\subsection{Paper selection}
Paper-wise, only articles with a publication date older than 2010 are concerned. This literature review sets this date because nudging is a rather new concept that first was introduced under this definition in 2009 (\cite{thaler_nudge:_2009}). To obtain relevant articles, a keyword-based search is conducted. The major keywords in this search are \textit{nudg* AND digital}. A full-text search is conducted on all journals. Because the term nudging is not always directly mentioned in the articles, additional keywords are added to the search query if the examined journal does not provide any necessary results with regards to the keyword \textit{nudg*}. Those additional keywords are \textit{decision, choice, consumer}. Overall 534 journal articles were identified in the initial search. Those articles mention the term nudge or match the described keywords. Articles are excluded based on several criteria. This concerns journal papers that only embody offline nudges, as well as articles that focus on the topic of persuasion. 
\\

In the end, the final concept matrix includes 37 research articles. Figure \ref{fig:method} shows the information flow and screening process. The complete list of articles is available in the appendix (Table \ref{table:articles}, \ref{table:articles-2} and \ref{table:articles-3})

\subsection{Analysis approach}
To guide the analysis, the research takes several questions into account. The structure of those questions is based on prior literature research. (\cite{alavi_review_1992}). On the one hand the research approach and field of use is evaluated to recognize trends in current research. On the other hand, major theories, concepts, heuristics, and biases are investigated to study the effect and implementation of digital nudges.
Concerning the in-depth analysis, a concept matrix codes the identified articles. To answer the underlying research questions, this paper inspects different categories of the relevant articles. Those categories are 
\begin{itemize}
\item General research information and metadata
\item Influence on choice architecture
\item Underlying concepts and theories
\end{itemize}

A complete version of the coding and concept matrix is available in the appendix (see table \ref{table:empirical-coding}, \ref{table:empirical-coding-2} and \ref{table:nonempirical-coding}).