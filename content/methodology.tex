\section{Methodology}



- Literature review in limited scope, not all papers and studies in the disciplines
- Part examined. Tried to be as representative as possible
- Research approach follows pattern in information systems literature reviews
	Identify, review, analyze existing literature
	Identify theoretical and methodological approaches
	Identify research gap withing literature
- Different variables important for designing efficient search strategy
- Not feasible to analyze all articles of digital nudges
- Concept across several streams of research
- Variables for journals and papers
- Journal-wise: reasonable to examine the papers with most influence in the domain
- Nudges topic with several research streams; papers spread across wide range
- To identify relevant research articles, identify top journals in domain
- Basket of 8 in AIS (Alavi 1992)
- Management and Marketing Journals from UT Dallas List
- VHB Rating for Decision Making, Consumer Research and Psychology Journals
- Conference Articles
- Paper Wise: More than 2 pages, younger than 2011, nudging is rather new concept
- Keyword based search (nudge, digital) ---> Additional keywords to make choice more relevant if a lot of articles were found
- Nudge is not directly mentioned always
- Papers that only speak about offline excluded
- Papers that speak about persuasion are excluded
- In the end XY articles found
- To guide analysis, several questions
	What is the kind of choice?
	What is the research approach?
	What major theories and concepts were used to study the effect of nudges
	What part of decision-making / choice architecture was influenced
- Prior research showed that these questions allow researchers to synthesize research field in information system discipline (Alavi, Webster ?!) --> REWRITE!
- First step in analysis is coding
- Different categories examined to answer underlying questions
- Those categories are...
- Full version of the coding is available in the appendix.

Sources:
- Webster 2002
- Alavi 1992

Quotes:
- Journal-wise variables are the journal domain and its rating. As suggested in existing literature, it is reasonable to not only look within the field of information systems research, but also outside (Webser 2002)

General content: 
-	Overall search strategy
-	Identification of relevant journals and why they’re used
-	Used keywords, queries and terms for the search
-	Classification and patterns of papers