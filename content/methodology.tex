\section{Methodology}

This literature review follows a systematic approach, that is well-established in the discipline of information systems (\cite{webster_analyzing_2002}). It was done in a limited scope which means that it does not cover all papers and studies of the subject. Rather, this literature review targets a qualitative subset of literature and thereby tries to be as representative as possible. The focus lies on publications of academic journals. To get a qualitative representation of the current research a journal-wise analysis is preferable to a database-based analysis. The overall approach follows a known pattern in information systems literature reviews (\cite{alavi_review_1992}).
\begin{enumerate}
\item Identifying, reviewing and analyzing existing literature in the field of digital nudges. This includes empirical, as well as non-empirical studies
\item Identifying theoretical and methodological approaches used to understand the use of nudges in consumer choice. This also includes the type of choices and the designed choice architecture.
\item Identifying a research gap within existing literature to guide future research.
\end{enumerate}

To realize this strategy, several variables are necessary to consider while identifying qualitative representative research articles. Digital nudging is a concept that spans across several fields of research. At the same time the understanding as well es implementation and studying differs widely. Therefore, this literature review aims to cooperate  different research streams to build a common ground, by identifying and analyzing the most representative research articles in the domain. Thereby search variables are set journal- and paper-wise. A graphic of the screening process is available in the appendix (Figure \ref{fig:method}).

\subsection{Journal selection}
Journal-wise variables are the journal domain and its rating. As suggested in existing literature, it is reasonable to not only look within the field of information systems research but also outside (\cite{webster_analyzing_2002}). It is reasonable to examine academic journals with the most influence in the research domain. As already mentioned, nudging is a subject of several research streams. This includes research from the area of information systems, management, marketing, behavioral economics, and psychology. Regarding research about information systems the \textit{AIS Basket of 8} provides a good source (\cite{alavi_review_1992}). This basket consists out of eight well-respected journals in the domain. After the AIS scholarly basket, academic journals about management and marketing were recorded in the research process. Thereby, the journal list of the \textit{UT Dallas} was taken as a reference point. Overall, the journal list of the UT Dallas contributes with twelve journals to the research pool. This paper also includes academic journals from the domains of behavioral economics, decision making and psychology (with regards to human decision making) in the research process, to gain further insights into the concept of nudges. The relevant publications are identified by the \textit{VHB} journal rating \textit{JOURQUAL3}\footnote{more information under \url{https://vhbonline.org/vhb4you/jourqual/vhb-jourqual-3/gesamtliste/}}. Journals with a rating of \textit{B} or better are taken into account for the research. To finalize the list of sources for the upcoming analysis, conference publication from the AIS pool with a VHB rating of \textit{B} or better were included, too. In total 36 journals were examined. A complete list of these journals is accessible in the appendix. %TODO

\subsection{Paper selection}
Paper-wise, only articles with a publication date older than 2010 are concerned. This literature review sets this date because nudging is a rather new concept that first was introduced under this definition in 2009 (\cite{thaler_nudge:_2009}). To obtain relevant articles, a keyword-based search is conducted. The major keywords in this search are \textit{nudg* AND digital}. A full-text search searches all journals. Because the term nudging is not always directly mentioned in the articles, additional keywords are added to the search query if the examined journal does not provide any necessary results with regards to the keyword \textit{nudg*}. Those additional keywords are \textit{decision, choice, consumer}. Overall 87 journal articles were found that mentioned the term nudge or matched the described keywords. To extract the most relevant sources, articles were excluded to based on several criteria. This concerns journal papers that only embody offline nudges. Such articles were excluded from the final article list, as well as articles that focus on the topic of persuasion and long-term behavior change. In the end, the final concept matrix evaluates 37 research articles. The complete list of articles is available in the appendix. %TODO

\subsection{Analysis approach}
To guide the analysis, the research takes several questions into account. The structure of those questions is based on prior literature research. (\cite{alavi_review_1992}).
\begin{itemize}
\item What is the type of choice?
\item What is the research approach?
\item What major theories, concepts, heuristics, and biases are used to study the effect of the evaluated nudge and how is the choice guided?
\item What part of the choice architecture is influenced?
\end{itemize}

Concerning the in-depth analysis, a concept matrix codes the extracted articles. To answer the underlying research questions, this paper inspects different categories of the relevant articles. Those categories are 
\begin{itemize}
\item General research information and metadata
\item Influence on choice architecture
\item Underlying concepts and theories
\end{itemize}

A complete version of the coding and concept matrix is available in the appendix. %TODO