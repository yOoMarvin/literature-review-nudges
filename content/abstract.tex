\sectionunnumbered{Abstract}
With the release of the book "Nudge" by Richard Thaler and Cass Sunstein, the concept of nudging gained interest in research. Nudging describes a way that alters people's decisions and behaviors predictably. Therefore, this concept has promising benefits for different target groups. One the hand, the business can increase conversions with precisely designed nudges and on the other hand users face easier decisions they have to make which overall can increase the user experience and - satisfaction. This paper builds a literature review of different research streams concerning nudges in digital environments. A qualitative representation of the current research in forms of academic journal articles is gathered, reviewed and analyzed.
This literature review sees a clear focus of nudging in the application area of consumer choice and an emphasis on empirical studies. The vast majority of research studies the effect of social-, as well as default nudges. To design those nudges, researchers consider only a few specific biases such as the status quo bias and social norms. Overall, current research provides only an isolated view on decision-making with the focus on the nudge itself. No research with regards to the overall user experience is identified. The results and limitations provide essential findings for future research in the area of digital nudging. This paper recommends a focus for research on complex decision domains like the finance and insurance sector, as well as an extended analysis of other heuristics and biases such as the middle-option bias and scarcity effect. Furthermore, the findings of this paper provide exciting insights for product designers and -managers to facilitate decision-making processes in digital applications.
