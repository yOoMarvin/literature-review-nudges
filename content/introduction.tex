\section{Introduction}
It is a typical Sunday afternoon. John is sitting on the couch, watching the match of his favorite soccer club on TV. On his lap, he is holding his tablet while browsing the internet. John is looking for a good travel deal for his upcoming trip to Bali with his girlfriend. On a news site, a prominent and bright advertisement catches his attention: \textit{Booking.com - From cozy country homes to funky city apartments}. That is precisely what John is looking for. He clicks the link and finds himself on a website full of amazing images of traveling people. Moreover, there is a search field, too. John enters his dream-destination, the travel time and clicks on "search." After some seconds a list of hotels shows up. The first one catches his eyes. A beautiful beach, a nice pool and cozy, big bedrooms. Perfect. He clicks on the details. However,  John is starting to become nervous. A bright, red piece of information is saying to him that this room has been booked three times in the last twelve hours. Also, there are only seven rooms left! His heart beats faster. He needs to get that deal! John clicks on the reservation button. He just has been nudged\footnote{A screenshot of the web page can be found in the appendix on figure \ref{fig:booking}}.

%In the last ten years, our lives became more and more digitized. We buy products in online shops, book our next trip and holidays on digital hotel platforms and even manage our finances with the help of our smartphones (\cite{schneider_digital_2018}). All those digital environments have one thing in common. Choices. All the time users are faced with choices they have to take,  even if people do not perceive it directly. There a lot of things digital (and also non-digital) environments that frame the whole choice process and therefore influence the decision-making through certain biases and heuristics (\cite{tversky_judgment_1974}). 

\cite{johnson_beyond_2012} states that "what is chosen often depends on the representation." This representation describes as the term of choice architecture, which should "alter people's behavior in a predictable way" (\cite{thaler_nudge:_2009}). In the age of digital transformation, digital environments are powerful tools where the choice architecture can be controlled in detail and therefore provide opportunities to influence user behavior in several ways with the help of user-interface design elements. This process is called "digital nudging" (\cite{weinmann_digital_2016}).

Digital nudging and the design of online choice architecture have recently gained interest in different research areas. Because of the complexity behind this concept, it is significant to understand how such nudges influence the decision-making of the user and how the cognitive biases behind this process are working. Especially in consumer choices, there are good and bad patterns of nudging when it comes to an ethical point of view (\cite{sunstein_nudging_2015}). To get a better understanding of how digital nudges influence consumer choice this paper presents a systematic literature review from the last ten years in a scientific manner. 

%This introduction is followed by a description of the conceptual and theoretical background to get a better understanding of the concept of digital nudging and the theories behind it. Afterward, the method and approach of the literature review are introduced to the reader. In the central part, the results are discussed where also a research gap is presented to give some suggestions for future research.

The goals of this paper are two-folded. The primary aim is to provide an overview of different research streams within the topic of digital nudges. The author focuses here on digital nudges in the area of consumer choice and their specific design elements.  Literature in this domain shall be gathered, reviewed and analyzed. Secondary, a recommendation for future research is derived from the analysis to advance research in this particular subject. Because of the multidisciplinary assortment of digital nudged, this paper contributes to several scientific domains. First of all, it is major implications for the area of information systems by showing areas with little research. Furthermore, the paper holds implications for the areas of marketing and consumer research as well as psychology and behavioral economics with regards to digital environments
