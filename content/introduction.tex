\section{Introduction}
In the last ten years, our lives became more and more digitized. We buy products in online shops, book our next trip and holidays on digital hotel platforms and even manage our finances with the help of our smartphones (\cite{schneider_digital_2018}). All those digital environments have one thing in common. Choices. All the time users are faced with choices they have to take,  even if people do not perceive it directly. There a lot of things digital (and also non-digital) environments that frame the whole choice process and therefore influence the decision-making through certain biases and heuristics (\cite{tversky_judgment_1974}). 

\cite{johnson_beyond_2012} states that "what is chosen often depends on the representation." This representation describes as the term of choice architecture, which should "alter people's behavior in a predictable way" (\cite{thaler_nudge:_2009}). In the age of digital transformation, digital environments are powerful tools where the choice architecture can be controlled in detail and therefore provide opportunities to influence user behavior in several ways with the help of user-interface design elements. This process is called "digital nudging" (\cite{weinmann_digital_2016}).

One example of such a digital nudge is present on the online travel platform \textit{booking.com}. Here users typically search for travel accommodations. The result page lists several hotel rooms. With the description of the room and the price comes a piece of information that this hotel was booked several times in the last 12 hours and that there are only a specific number of rooms available. In bright, red color this should create a scarcity in the mind of the user, to perceive the good as more valuable. Through this digital nudge, the likelihood of reservation in this particular hotel is increased\footnote{A screenshot of the web page can be found in the appendix on figure \ref{fig:booking}}.

Digital nudging and the design of online choice architecture have recently gained interest in different research areas. Because of the complexity behind this concept, it is significant to understand how such nudges influence the decision-making of the user and how the cognitive biases behind this process are working. Especially in consumer choices, there are good and bad patterns of nudging when it comes to an ethical point of view (\cite{sunstein_nudging_2015}). To get a better understanding of how digital nudges influence consumer choice this paper presents a systematic literature review from the last ten years in a scientific manner. 

%This introduction is followed by a description of the conceptual and theoretical background to get a better understanding of the concept of digital nudging and the theories behind it. Afterward, the method and approach of the literature review are introduced to the reader. In the central part, the results are discussed where also a research gap is presented to give some suggestions for future research.


The goals of this paper are versatile. The primary aim is to provide an overview of different research streams within the topic of digital nudges. The author focuses here on digital nudges in the area of consumer choice and their specific design elements.  Literature in this domain shall be gathered, reviewed and analyzed. Secondary, a recommendation for future research is derived from the analysis to advance research in this particular subject. Because of the multidisciplinary assortment of digital nudged, this paper contributes to several scientific domains. First of all, it is major implications for the area of information systems by showing areas with little research. Furthermore, the paper holds implications for the areas of marketing and consumer research as well as psychology and behavioral economics with regards to digital environments
