\section{Introduction}
In the last ten years, our lives became more and more digitized. We buy products in online shops, book our next trip and holidays on digital hotel platforms and even manage our finances with the help of our smartphones (\cite{schneider_digital_2018}). All those digital environments have one thing in common. Choices. All the time users are faced with choices they have to take. 

-	A lot of things influence those choices
-	What is chosen depends on the representation (Johnson in Weinmann paper)
-	This is described as choice architecture
-	Digital environments are powerful tools where choice architecture can be controlled in detail
-	This provides opportunities where user decisions can be influenced in several ways
-	This is called nudge
-	Example nudge: booking.com
-	Digital nudging and design of online choice architecture became more and more popular in research
-	Because of the complexity it is important to understand how nudges influence decisions of the user
-	Especially in consumer choices there are good and bad use cases (patterns)
-	To get a better understanding of how digital nudges influence consumer choice
-	This paper represents a systematic literature review of studies from the last ten years where offline- as well as online nudges are investigated in scientific manner
-	Specifically, this means that the paper structures as follows
-	Theoretical background to understand the domain of digital nudging and the theories and concepts behind it
-	Afterwards method and approach is introduced
-	Keywords, databases, journals
-	Results are discussed
-	Gap in finance and insurance sector is identified and discussed in the next part
-	In the end limitations, conclusions, implications for future research
