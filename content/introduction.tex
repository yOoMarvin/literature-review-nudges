\section{Introduction}
In the last ten years, our lives became more and more digitized. We buy products in online shops, book our next trip and holidays on digital hotel platforms and even manage our finances with the help of our smartphones (\cite{schneider_digital_2018}). All those digital environments have one thing in common. Choices. All the time users are faced with choices they have to take. And even if people don't perceive it directly. There a lot of things digital (and also non-digital) environments that frame the whole choice process and therefore influence the decision-making through certain biases and heuristics (\cite{tversky_judgment_1974}). 

\cite{johnson_beyond_2012} states that "what is chosen often depends on the representation". This representation is often described as the term of choice architecture, which should "alter people's behavior in a predictable way" \cite{thaler_nudge:_2009}. In the age of digital transformation, digital environments are powerful tools where the choice architecture can be controlled in detail and therefore provide opportunities where user behavior can be influence in several ways with the help of user-interface design elements. This process is called "digital nudging" (\cite{weinmann_digital_2016}).

One example for such a digital nudge can be found on the online travel platform \textit{booking.com}. Here users typically search for travel accommodations. On the results pages, several hotel rooms are listed. With the description of the room and the price comes an information that this hotel was booked several times in the last 12 hours and that are only a specific number of rooms free. In bright, red color this should create a scarcity in the mind of the user, so the good is perceived as more valuable. Through this digital nudge, the likelihood of reservation in this particular hotel is increased\footnote{A screenshot of the web page can be found in the appendix on \ref{fig:booking}}.

Digital nudging and the design of online choice architecture have recently gained interest in research. Because of the complexity behind this concept it is important to understand how such nudges influence the decision-making of the user and how the cognitive biases behind it work. Especially in consumer choices there are good and bad patterns of nudging when it comes to an ethical point of view (\cite{sunstein_nudging_2015}).

To get a better understanding of how digital nudges influence consumer choice this paper presents a systematic literature review of studies from the last ten years where offline- as well as online nudges are investigated in scientific manner. This introduction is followed by description of the conceptual and theoretical background to get a better understanding of the concept of digital nudging and the theories behind it. Afterwards the method and approach of the literature review is introduced to the reader. In the main part the results are discussed where also a research gap is presented to give some suggestions for future research.
