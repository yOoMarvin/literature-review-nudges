\section{Conclusion}

 \subsection{Summary of findings}
The results of this literature review are manifold. A vast majority of identified research articles studies nudging in the consumer choice area. Additionally, the superior part deals with empirical research, which provides more precise information and contributions to nudging research. Empirical studies build a useful way of studying the effect of nudges and furnish ways to understand the use of them in a practical environment. With a clear focus on laboratory experiments, researchers limit on an isolated view on the decision-making process and the nudge implementation. Furthermore, it is important to mention, that also other factors than the nudge itself influence the decision environment. Other variables that do not directly have an impact on the decision should also be taken into account (\cite{biswas_shining_2017}).
\\

Considering the type of digital nudges, research articles lay a clear focus on default effects and social nudges with the status quo bias and social norms as most used heuristics and biases. Moreover, the current research targets mainly the choice architecture categories of decision information and decision structure. The category of decision assistance is less pronounced. One of the underlying reasons for that is the psychological background and complex implementation of nudges. Decision assistance often goes beyond the boundaries of digital environments so that a digital nudge affects a decision in a non-digital environment. 
 
\subsection{Limitations}
It is important to notice that not every research article directly mentions the concept of nudging. One possible explanation for this is the novelty of this concept. With the first research in 2010, nudging theory is known for less than ten years. Additionally, nudging has its roots in the area of behavioral economics. Other research streams might study the same concept but do not label it as nudging directly. If nudging research continues, this problem will fade with the broader adoption of the term "nudging" across research streams. At the moment many research articles, specifically about digital nudging, are in the review process and will probably be released in the near future. Those articles in review are not part of this literature review.
\\

Moreover, there is only a limited amount of articles that mentions specific interface design elements of digital nudges. Design elements are the building stones of every user interface and therefore have a significant influence on the choice architecture presented to the user. Unfortunately, those design elements are rarely specified in current studies. Studies with a focus on digital nudging barely include detailed descriptions of the digital environment and the user interface of the experiment. Such additional information can have a significant impact on the understanding of digital nudges.
\\
 
\subsection{Recommendations for future research}
Based on the limitations and findings of this literature review, one can derive several recommendations for future research. One of those recommendations is the use of further heuristics. As described in the review of the literature, no research article uses the scarcity effect, as well as the middle-option bias. Both techniques successfully support nudging in non-digital environments. Therefore, a transfer to digital choice environments might provide promising results.
\\
 
Another implication for future research is the connection to the domain of Human-Computer Interaction. Here, further research on how design elements influence digital nudges can be conducted. This can provide useful information for product designers and managers, to build efficient choice architectures. 
 \\
 
Finally, future research can place more emphasis on other domains than consumer choice, especially domains where complex decisions are taken. Such a domain, for example, is the finance and insurance sector. Here, users face complex decisions without being domain experts. In the past, a lot of financial services and insurance contracts were sold with the help of a consultant from the sales department. Nowadays, more and more self-services allow users to act by themselves and independently. %TODO Source?
Such an example is the online bank N26 that provides full banking functionalities and services without a physical bank in the background. The only interfaces between bank and user are the mobile app and the web dashboard. In finance and insurance, decisions typically have a long term effect and usually are locked-in for an extended time. That is why it is critical to make good decisions and to provide an easy to understand choice architecture for users. Besides the likely possibilities, nudging is very critical in this domain, because of the importance of the ethical aspect of nudging. If done wrong, a company can quickly lose the trust of users and conversions might decrease. With a field experiment in the finance and insurance sector, future research can study the possible performance of digital nudges and provide useful insights for nudging theory as well as for product designer and managers.