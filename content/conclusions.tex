\section{Conclusion}

 \subsection{Summary of Findings}
The results of this literature review are multifaceted. As shown previously, the vast majority of identifed research articles studies nudging in the consumer choice area. Additionally, the superior part deals with empirical research. In summary the studies with a focus on provider overall better information and contributions to the concept of nudging. Empirical studies usually build a useful way of studying the effect of nudges and furnish ways to understand the use of them. With a clear focus on laboratory experiments, researchers limit on an isolated view on the decision-making process and the nudge implementation. Furthermore, it is important to mention, that also other factors than the nudge itself have important influence on the decision environment, so that other variables that do not directly have an impact on the decision should also be taken into account. \cite{dong_cueing_2019}.

Considering the type of digital nudges, research articles lay a clear focus on default effect and social nudges. With the status quo bias and the norms as most used heuristics and biases in the nudge design, support this observation. Moreover, the current research targets mainly the choice architecture categories of decision information and decision structure. The category of decision assistance is less pronounced. One of the underlying reasons for that might be the psychological background and complex implementation of nudges. Decision assistance often goes beyond the boundaries of digital environments. So that a digital nudge should affect a decision in a non digital environment. 
 
 \subsection{Limitations}
Despite the promising results and insights concerning digital nudging, the identified research papers, as well es the overall literature review have some limitations.
First of all it is important to mention that not every of the of the research articles do not directly  specified the studied concept as nudging. One possible explanation for this is the novelty of the concept of nudging. With the first research in the late 2000's, nudging itself is known less than ten years. Additionally, nudging has its roots in the area of behavioral economics. Other research streams might study the same concept but do not label it as nudge directly. If nudging research continues, this problem will fade away with a broader adoption of the term "nudging" across research streams. At the moment a number of research articles, specifically about digital nudging, is in the review process and will probably be released in the near future. Those articles in review are not part of this literature review. 

Moreover, there is only a limited amount of articles that mentions specific interface design elements of digital nudges. Design elements are the building stones of every user interface and therefore have great influence on the choice architecture presented to the user. Unfortunately, those design elements are rarely specified in current studies. Studies with a focus on digital nudges barely included screenshots of the digital environment and the user interface of the experiment. Such additional information can have great impact on the understanding of digital nudges.

In the end, this literature review was carried out in a limited time and scope. It is important to mention that not every article in the domain is included within this literature review. Moreover, a qualitative part of the literature is identified trying to be as representative as possible.
 
 \subsection{Recommendations for future research}
 
 Based on the limitations and findings of this literature review, one can derive several recommendations for future research.
 One of those recommendations includes the use of other heuristics. As described in the review of literature, no research article uses the scarcity effect, as well as the middle-option bias. Both techniques successfully support nudging in non-digital environments. Therefore, a transfer to digital choice environments might provide promising results.
 
 Another implication for future research is the connection to the domain of HCI. Here, further research on how design elements influence digital nudges can be conducted. This can provide useful information for product designers and managers, to build efficient choice architectures. 
 
 Finally future research should place more emphasis on other domains than consumer choice. Especially domains where complex decisions are taken. Such a domain is the finance and insurance sector. Here, users face complex decisions without being having the knowledge of domain experts. In the past, a lot of financial services and insurances were sold with the help of a consultant of the sales department. Nowadays, more and more self-services allow users to act by themselves and independently. Such an example is the online bank N26, that provides full banking functionalities and services without a physical bank in the background. The only interfaces between bank and user are the mobile app and the web dashboard. In finance and insurance, decisions typically have a long term effect and usually are locked-in for a long period of time. That's why it is very important to make good decisions and to provide an easy choice architecture for users. Besides the promising possibilities, nudging is very critical in this domain, because of the importance of the ethical aspect of nudging. If done wrong, a company can easily loose trust of users and conversions might go down. 