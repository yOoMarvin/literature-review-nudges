\section{Conclusion}
In the end of the thesis all results will be summarized and critically discussed. The goal is to identify a research gap and to give recommendations for future research that would advance the topic.


 \subsection{Summary of Findings}
 - multifacet findings
- vast majority of research in consumer area
- empirical research
- provides better information, more useful to study the effect of nudges and understand them
- Focus on lab experiments
- Isolated view on decision-making process and nudge implementation
- Other factors are also important (outside variables that influence indirectly) --> Ambience (paper mit pitch)
- Clear focus on default effects and social nudges
- Most used heuristics and biases in nudge design
- Focus on decision information and decision structure
- decision assistance complex, very psychological
- goes over the bounderies of digital environment
 
 \subsection{Limitations}
 - Not every study talked directly about digital nudging or nudging itself...
 - Not everyone calls it nudge
 - Limited time and scope
 - New topic
 - Lot of articles currently in review
 - No mention of design elements (only few)
 
 \subsection{Recommendations for future research}
 

- Recommendations
- use other heuristics --> Scarcity, middle-option
- More focus on design elements --> HCI perspective
- More emphasis in complex domains and complex decisions
- Finance and Insurance sector
- Decision complex for users without expert knowledge
- More and more self services, like online bank N26
- decisions have a long term effect, locked-in for a long time
- That's why it's very important to make good decisions
- But, nudging very critical here. Ethical aspect important
- Easily can loose trust of user if nudge is to "aufdringlich"