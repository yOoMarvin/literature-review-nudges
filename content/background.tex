\section{Conceptional Background}

\subsection{ Birth of Nudges}
With the release of the book "Nudge" in 2009, Thaler and Sunstein have laid the foundation stone for the concept of nudging. This concept was primarily a subject of research in behavioral economics. Because of the multifaceted meaning of the word \textit{nudging}, a consistent understanding is essential. Further on, this paper uses the central definition of nudges from \cite{thaler_nudge:_2009}:
\begin{center}
\textit{"A nudge [...] is any aspect of the choice architecture that alters people's behavior in a predictable way without forbidding any options or significantly changing their economic incentives."}
\end{center}

One central aspect of this definition is the economic incentive of the consumer, which should not be changed. This fundamental thought is the basis of a concept called \textit{libertarian paternalism}. In this concept, choices are influenced in a way to make them easy for people and aligning them with their interests. One example of that would be "putting the fruit at eye level." However, banning the food would not be a nudge. (\cite{thaler_nudge:_2009}). This principle is the foundation of nudges for a good reason. Influencing people's behavior can simply be exploited. So, the ethical viewpoint on nudges should always be kept in mind when implementing and using them to guide customer choices (\cite{sunstein_nudging_2015}).

The underlying foundation for nudging the cognitive limitation of human brains. Because the human brain only has a limited capacity to store and process information, the consumer often feels subconsciously overloaded. This results in greater difficulty and complexity when it comes to decisions and cognitively demanding tasks (\cite{broniarczyk_decision_2014}). Therefore "many decisions are based on beliefs concerning the likelihood of uncertain events (\cite{tversky_judgment_1974}). Based on this assumption \cite{tversky_judgment_1974} formulated three heuristics and several biases that build the underlying foundation of human decision making. Those heuristics and biases can also be found acting as a guideline in the world of digital nudges.
% Add heuristics here? Later in digital nudges? Schneider 2018?

Besides the cognitive foundation of decision making, also the principles of nudges play a major role in their application and implementation. Overall, there are five general principles of nudging (based on \cite{thaler_choice_2010})
\paragraph{Incentive}
Those kinds of nudges aim to make incentives more salient to increase the effectiveness of the nudge. The focus lays on the motivation behind the decision. The nudge should always search for the right motivation for the right people. This motivation goes beyond monetary and material incentives.
\paragraph{Understanding mapping}
Making the consequence of a choice clear is an essential part of easing the decision-making. Mainly, this concerns complex information that is difficult to evaluate. For example, the number of megapixels of a camera. Frequently, customers cannot evaluate this information directly and only compare on a single number. A rational mapping would be to display the maximum printable size of a taken picture. This way, the product attribute can be compared efficiently.
\paragraph{Defaults}
The pre-selection of certain information has enormous power. By changing the default option, consumers are more likely to choose that near to the selected default or even is the default. One prominent example of such a nudge is the question if people want to consent to be an organ donor. Simply by changing the default option, in this case, can nearly double the percentage of organ donors (\cite{johnson_defaults_2003}). 
\paragraph{Giving feedback}
By giving feedback during the decision-making, people can evaluate their performance and estimate the output or consequences of the decisions they face. Such an example can be found in an experiment for pre-ordering lunch in a school. Students arrange their lunch with different kind of foods. According to this arrangement they receive feedback about how balanced and healthy their food compilation is. Only based on this feedback, students selected significantly more fruits and vegetables in their meals (\cite{miller_effects_2016}).
\paragraph{Expecting error}
Precisely because of the underlying complexity of the decision-making process, it is necessary to expect errors to be made. Such errors should be taken into account when designing a decision, and the environment should be as forgiving as possible. In complex choice environments, such as the food of healthy and balanced food, many people make mistakes. By giving direct feedback on those errors and providing information on how to improve the performance, this decision can be made easier (\cite{guthrie_nudging_2015}).
\paragraph{Structure complex choices}
Another difficult task in decision-making is to compare different product alternatives. By listing all attributes, people can evaluate trade-offs and make better decisions, based on their interests. In a field experiment, researches evaluated the effect of such a nudge in a bar, when it comes to craft beer choice. By listing product more product attributes that naturally describe the taste, people could decide easier what they want to order (\cite{malone_excessive_2017}).


\subsection{(Online) Choice Architectures}
The concept of nudges builds on the assumption that decisions are made in choice architectures, which are designed by choice architects (\cite{thaler_nudge:_2009}). In this case, the parallel to a "real" architect of a building is not far-fetched. \cite{johnson_beyond_2012} describes the power of such choice architects and how choice architects guide people's choices like other architects guide behavior through the design of the "placement of doors, hallways, staircases, and bathrooms. Just like in a hotel or building, "there is no neutral architecture" (\cite{johnson_beyond_2012}) for choices. Even small things like a default choice affect the decision which is made by the user. The mobile payment app Square, for example, nudges people into giving tips only by setting a default value. This way, customers actively must select a "no tipping" option if they do not want to give a tip (\cite{weinmann_digital_2016}). "Because advances in technology and the user of the Internet also provide new ways of finding, creating and exchanging information [...]" (\cite{broniarczyk_decision_2014}) people automatically shifted a majority of their decisions in the online or digital world. However, those digital environments are not less complex. Just like in offline environments, there is no neutral way to present choices. Therefore, any user interface can be viewed as a digital choice environment (\cite{schneider_digital_2018}). This ranges from the positioning of elements, the colors in the interface, the language, even the design elements themselves and beyond.

To get a better understanding of how such choice architectures can be built up and what elements are available,  \cite{munscher_review_2016} created a taxonomy of choice architecture categories and their techniques. Overall, there are three major categories with several associated techniques. 
\paragraph{Decision information}
The first level of choice architectures targets the "presentation of decision-relevant information" (\cite{munscher_review_2016}). One important aspect is that this category only includes the presentation and no altering of the options itself. Techniques for that choice architecture category are the translation of information, visibility of information and the providence of social reference points
\paragraph{Decision structure}
Secondly, choice architects directly modify the available options of choice itself. This includes techniques like choice defaults, the related effort and consequences of an option and also the range of composition and options.
\paragraph{Decision assistance}
Lastly, choices can be designed in a way that consumers follow their intentions. Techniques for such assistance can be the fostering of a commitment or by providing reminders of the preferred behavior.

\subsection{Nudging became digital}
Because various choices we take today "involve some form of information technology" (\cite{johnson_beyond_2012}), the concept of nudging recently gains interest in research of different disciplines. Thereby, the underlying concepts of "offline" nudges are transferred and adapted in digital environments. The result is digital nudges. According to \cite{weinmann_digital_2016} digital nudges are defined as follows:
\begin{center}
\textit{Digital nudging is the use of user-interface design elements to guide people's behaviors in digital choice environments.}
\end{center}
Just like in the offline and analog environments, digital environments face multiple sources of decision difficulty such as task complexity, information load, information uncertainty, conflicts, emotional difficulty and preference uncertainty (\cite{broniarczyk_decision_2014}). To face those challenges in digital environments, the use of cognitive heuristics and biases can act as a baseline to design digital nudges. Different user-interface design elements facilitate different nudges. Table \ref{table:schneider} gives an overview of the different biases, in which way they influence decision-making and how those are translated to specific design elements.

\begin{table}[h!]
\begin{tabular}{l|l}
\textbf{Heuristic / Bias}  & \textbf{Example Design elements and mechanisms}                                                                                                                                                \\ \hline
Status quo bias            & \begin{tabular}[c]{@{}l@{}}- Radio buttons\\ - Checkboxes\\ - Dropdown menus\\ - Sliders with default position\\ - Pre-filled inputs\end{tabular}                                              \\ \hline
Decoy effect               & \begin{tabular}[c]{@{}l@{}}Presentation of options in: \\ - Radio buttons\\ - Checkboxes\\ - Dropdown menus\end{tabular}                                                                       \\ \hline
Primacy and recency effect & \begin{tabular}[c]{@{}l@{}}Positioning of elements \\ (earlier or later)\end{tabular}                                                                                                          \\ \hline
Middle-option bias         & \begin{tabular}[c]{@{}l@{}}- Addition of higher- and lower-price alternatives \\ around the preferred option.\\ - Ordering of alternatives.\\ - Modification of the option scale.\end{tabular} \\ \hline
Anchoring and adjustments  & \begin{tabular}[c]{@{}l@{}}- Variation of slider endpoints.\\ - Use of default slider position.\\ - Predefined values in text boxes for quantities.\end{tabular}                               \\ \hline
Norms (moral / social)     & \begin{tabular}[c]{@{}l@{}}- Display of popularity (social norms).\\ - Display of honesty codes (moral norms)\end{tabular}                                                                     \\ \hline
Scarcity effect            & \begin{tabular}[c]{@{}l@{}}- Use of default slider position.\\ - Language and displaying additional information \\ about quantity and availability\end{tabular}                               
\end{tabular}
\caption{Heuristics and Design elements of digital nudges (based on \cite{schneider_digital_2018})}
\label{table:schneider}
\end{table}

% Add description of the table?

Even though nudges aim to influence behavior in digital environments, they should not be mistaken with persuasion. A persuasion is instead a form of human communication, that is also used in technology. The goal of this technique is also to influence user behavior, but more persistently, so that underlying attitudes are influenced (\cite{oinas-kukkonen_persuasive_2009}). Although both concepts share similarities, this paper solely focuses on digital nudges and the decision-making process. An ongoing influence on underlying behavior is still possible, but not directly part of a nudge and therefore not further evaluated.

% Types of Choices?
% Design Elements?
