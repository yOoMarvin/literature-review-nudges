\section{Conceptual Background}

\subsection{ The birth of nudging}
With the release of the book \textit{Nudge} in 2009, Thaler and Sunstein have laid the foundation for the concept of nudging. This concept was primarily a subject of research in behavioral economics. Because of the multifaceted meaning of the word \textit{nudging}, a consistent understanding is essential. This paper uses the central definition of nudges from Thaler and Sunstein (\citeyear[p.6]{thaler_nudge:_2009}): \textit{"A nudge [...] is any aspect of the choice architecture that alters people's behavior in a predictable way without forbidding any options or significantly changing their economic incentives."}
One central aspect of this definition is the economic incentive of the consumer, which should not be changed. This fundamental thought is the basis of a concept called libertarian paternalism. In this concept, choices are influenced in a way to make them easy for people and aligning them with their interests. One example of that would be "putting the fruit at eye level" (\cite[p.6]{thaler_nudge:_2009}). However, banning the food would not be a nudge. Since influencing people's behavior can simply be exploited, the ethical viewpoint on nudges should always be kept in mind. (\cite{sunstein_nudging_2015}).
\\

Because the human brain only has a limited capacity to store and process information, the consumer often feels subconsciously overloaded. This overloading is called cognitive limitation, which results in greater difficulty and complexity when it comes to decisions and cognitively demanding tasks (\cite{broniarczyk_decision_2014}). Therefore "many decisions are based on beliefs concerning the likelihood of uncertain events" (\cite[p.1124]{tversky_judgment_1974}). Based on this assumption Tversky and Kahneman (\citeyear{tversky_judgment_1974}) formulated three heuristics and several biases that build the foundation of human decision making. Those heuristics and biases act as a psychological guideline in digital nudging. Table \ref{table:schneider} gives an overview of the most used heuristics and biases in nudging.
\\

Besides the cognitive foundation of decision making, there are five general principles of nudging (based on Thaler, Sunstein and Balz \citeyear{thaler_choice_2010})
\paragraph{Incentive}
Those kinds of nudges aim to make incentives more salient to increase the effectiveness of the nudge. The focus lays on the motivation behind the decision. The nudge should match the users' motivation. A motivation in nudges goes beyond monetary and material incentives.
\paragraph{Understanding mapping}
Making the consequence of a choice clear is an essential part of easing the decision-making process. Mainly, this concerns complex information that is difficult to evaluate, for example, the number of megapixels of a camera. Frequently, customers cannot evaluate this information directly, based on a single number. A rational mapping could be to display the maximum printable size of a taken picture (\cite{weinmann_digital_2016}). This way, the product attribute can be compared efficiently.
\paragraph{Defaults}
The pre-selection of certain information has an enormous effect. By changing the default option, consumers are more likely to choose an option near to the selected default or even the default itself. One prominent example of such a nudge is the question if people want to consent to be an organ donor. Simply by changing the default option, in this case, can nearly double the percentage of organ donors (\cite{johnson_defaults_2003}). 
\paragraph{Giving feedback}
By giving feedback during the decision-making process, people can evaluate their performance and estimate the output of the decision. Such an example can be found in an experiment for pre-ordering lunch in a school. Students arrange their lunch with different kind of foods. According to this arrangement they receive feedback about how balanced and healthy their food compilation is. Only based on this feedback, students selected significantly more fruits and vegetables in their meals (\cite{miller_effects_2016}).
\paragraph{Expecting error}
Precisely because of the underlying complexity of the decision-making process, it is necessary to expect errors to be made. Such errors should be taken into account when designing a decision, and the environment should be as forgiving as possible. That is why customers first have to remove their credit card from a cash machine before they can obtain the money. In this way, the system makes sure that users do not make serious (\cite{weinmann_digital_2016}).
\paragraph{Structure complex choices}
A difficult task in decision-making is to compare different product alternatives. By listing all attributes, people can evaluate trade-offs and make better decisions, based on their interests. In a field experiment, researches evaluated the effect of such a nudge in a bar regarding craft beer choice. By listing more product attributes that naturally describe the taste, people could decide more easily what they want to order (\cite{malone_excessive_2017}).


\subsection{(Online) Choice Architectures}
The concept of nudging builds on the assumption that decisions are made in choice architectures, which are designed by choice architects (\cite{thaler_nudge:_2009}). In this case, the parallel to a "real" architect of a building is not far-fetched. Johnson et al. (\cite{johnson_beyond_2012}) describe the power of such choice architects and how they guide people's choices like other architects guide behavior through the design of the "placement of doors, hallways, staircases, and bathrooms. Just like in a hotel or building, there is no neutral architecture" (\cite[p.488]{johnson_beyond_2012}) for choices. Even small things like a default choice affect decision-making. The mobile payment app Square, for example, nudges people into giving tips only by setting a default value. This way, customers actively must select a \textit{no tipping} option if they do not want to give a tip (\cite{weinmann_digital_2016}). "Because advances in technology and the user of the Internet also provide new ways of finding, creating and exchanging information [...]" (\cite[p.609]{broniarczyk_decision_2014}) people automatically shift a majority of their decisions to the online or digital world. However, digital environments are more complicated. Just like in non-digital environments, there is no neutral way to present choices. Therefore, any user interface can be viewed as a digital choice environment (\cite{schneider_digital_2018}). This ranges from the positioning of elements to the colors in the interface, the language and even the design elements themselves and beyond.
To get a better understanding of how such choice architectures can be built and which elements are available, Münscher et al. (\citeyear{munscher_review_2016}) created a taxonomy of choice architecture categories and their techniques. Overall, there are three major categories with several associated techniques. 
\paragraph{Decision information}
The first level of choice architectures targets the "presentation of decision-relevant information" (\cite[p.514]{munscher_review_2016}). One important aspect is that this category only includes the presentation and no altering of the options itself. Techniques for that choice architecture category are the translation of information, visibility of information and the providence of social reference points.
\paragraph{Decision structure}
Secondly, choice architects directly modify the available options of choice itself. This modification includes techniques like choice defaults, the related effort and consequences of an option and also the range of composition and options.
\paragraph{Decision assistance}
Lastly, choices can be designed in such a way that consumers follow their intentions. Techniques for such assistance can be the fostering of a commitment or by providing reminders of the preferred behavior.

\subsection{Nudging is becoming digital}
Because various choices we take today "involve some form of information technology" (\cite[p.490]{johnson_beyond_2012}), the concept of nudging recently gained interest in research of different disciplines. Thereby, the underlying concepts of non-digital nudges are transferred and adapted in digital environments. According to Weinmann et al.  (\citeyear{weinmann_digital_2016}), digital nudges are defined as follows:
\textit{"Digital nudging is the use of user-interface design elements to guide people's behaviors in digital choice environments"} (\cite[p.433]{weinmann_digital_2016}).
\\

Digital environments face multiple sources of decision difficulty such as task complexity, information load, information uncertainty, conflicts, emotional difficulty and preference uncertainty (\cite{broniarczyk_decision_2014}). To face those challenges, the use of cognitive heuristics and biases can act as a baseline to design digital nudges. Different user-interface design elements induce different nudges. Table \ref{table:schneider} in the appendix gives an overview of the different biases, in which way they influence decision-making and how those are translated to specific design elements.
\\

Even though nudges aim to influence behavior in digital environments, they should not be mistaken with persuasion. A persuasion is instead a form of human communication, that is also used in technology. The goal of this technique is also to influence user behavior, but more persistently, so that underlying attitudes are influenced (\cite{oinas-kukkonen_persuasive_2009}). Although both concepts share similarities, this paper solely focuses on digital nudges and the decision-making process. An ongoing influence on underlying behavior is still possible, but not directly part of a nudge and therefore not further evaluated in this literature review.

% Types of Choices?
% Design Elements?
