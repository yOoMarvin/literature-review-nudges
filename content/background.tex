\section{Conceptional Background}

\subsection{ Birth of Nudges}
With the release of the book "Nudge" in 2009, Thaler and Sunstein have laid the foundation stone for the concept of nudging. This concept was primarily a subject of research in behavioral economics. Because of the multifaceted meaning of the word \textit{nudging}, a consistent understanding is essential. Further on, this paper uses the central definition of nudges from \cite{thaler_nudge:_2009}:
\begin{center}
\textit{"A nudge [...] is any aspect of the choice architecture that alters people's behavior in a predictable way without forbidding any options or significantly changing their economic incentives."}
\end{center}

One central aspect of this definition is economic incentive of the consumer, which should not be changed. This fundamental thought is the basis of a concept called \textit{libertarian paternalism}. In this concept choices are influenced in a way to make them easy for people and aligning them with their interests. One example for that would be "putting the fruit at eye level". But banning the food would not be a nudge. (\cite{thaler_nudge:_2009}). This principle is the foundation of nudges for a good reason. Influencing people's behavior can simply be exploited. So, the ethical viewpoint on nudges should always be kept in mind when implementing and using them to guide customer choices (\cite{sunstein_nudging_2015}).

The underlying foundation for nudging cognitive limitation of human brains. Because the human brain only has a limited capacity to store and process information, consumer often feel subconsciously overloaded. This results in greater difficulty and complexity when in comes to decisions and cognitive demanding tasks (\cite{broniarczyk_decision_2014}). Therefore "many decisions are based on beliefs concerning the likelihood of uncertain events (\cite{tversky_judgment_1974}). Based on this assumption \cite{tversky_judgment_1974} formulated three heuristics and several biases that build the underlying foundation of human decision making. Those heuristics and biases can also be found acting as a guideline in the world of digital nudges.
% Add heuristics here? Later in digital nudges? Schneider 2018 ?

Besides the cognitive foundation of decision making, also the principles of nudges play a major role regarding their application and implementation. Overall, there are five general principles of nudging (based on \cite{thaler_choice_2010})
\paragraph{Incentive}
\paragraph{Understanding mapping}
\paragraph{Defaults}
\paragraph{Giving Feedback}
\paragraph{Expecting Error}
\paragraph{Structure complex choices}




\subsection{Online Choice Architectures}
The concept of nudges builds on the assumption that decisions are made in choice architectures, which are designed by choice architects (\cite{thaler_nudge:_2009}). In this case, the parallel to a "real" architect of a building is not far-fetched. \cite{johnson_beyond_2012} describes the power of such choice architects and how choice architects guide people's choices like other architects guide behavior through the design of the "placement of doors, hallways, staircases and bathrooms.

\subsection{Nudging became digital}
Digital nudges have the same underlying concepts as "offline" nudges. One major difference is the structure of them. 
Definition...
- digital nudging
- types of choices
- Offline focus – but more and more adoption in IS fields
- Forms of Digital Nudging (weinmann / thaler 2010 => Tabelle)


\subsection{Nudging versus Persuasion ?}



Quotes
- More and more of the choices we make involve the use of some form of information technology. This technology may be introduced to assist in the choice task (Johnson 2012
- Digital nudging is the use of user-interface design elements to guide people's behaviors in digital choice environments
- Choice Architects can make major improvements to the lives of others by designing user-friendly environments.
- As in offline environments, online environments offer no neutral way to present choices. Any user interface from organizational website to mobile app, can thus be viewed as a digital choice environment (Schneider 2018)
- Advances in technology and the use of the Internet also provide consumers new ways of finding, creating and exchanging information for choice (Broniarczyk 2014)
- Sources of decision difficulty: Task complexity, Information load, Information uncertainty, Conflict, emotional difficulty, preference uncertainty (Broniarczyk 2014)
- There is no neutral architecture (Johnson 2012) --> Even defaults effect user choice --> Square example by Weinmann 2016
